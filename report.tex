\documentclass[12pt,twoside]{article}

\usepackage{amsfonts}
\usepackage{amsmath}
\usepackage{microtype} %omitting bad boxes
\usepackage{multirow}
\usepackage{caption}
\usepackage{subcaption}
\usepackage{graphicx}
\usepackage{color}
\usepackage{epsfig}
\usepackage{graphicx}
\usepackage{rotating}
\usepackage{amssymb}
\usepackage{listings}
\usepackage{enumitem, color, amssymb}
\usepackage[dvipsnames]{xcolor}
\usepackage{pgfplots}
\usepackage{cancel}
\usepackage{hyperref}
\hypersetup{colorlinks = true, linkcolor = black, citecolor = black}
\pgfplotsset{compat=1.12}
\usetikzlibrary{decorations.markings}

\newcommand*{\COMMENTSBILL}{}%
\ifdefined\COMMENTSBILL
\newcommand{\commentbill}[1]{ {\color{red} #1}}% 
\else
\newcommand{\commentbill}[1]{}
\fi 

\newcommand*{\COMMENTSRAPISCAN}{}%
\ifdefined\COMMENTSRAPISCAN
\newcommand{\commentrapiscan}[1]{ {\color{blue} #1}}% 
\else
\newcommand{\commentrapiscan}[1]{}
\fi 

\title{Scatter tomography in threat detection \\ \vspace{0.1in} \small{EPSRC IAA 211 Lionheart Proof of Concept Scheme}}
\author{Naeem Mustaq Desai}


\begin{document}

\maketitle

Although the start date on paper for this project was 1st January, work regarding this project actually began by late January after my corrections for my PhD were completed. 
Here one can find a summary of the work done in section \textit{Progress} and a mention of next weeks work can be found in the section \textit{Next steps}. Furthermore there are some queries which can be found in
the \textit{Questions} section. If you want to make any suggestions macros have been added in the .tex file. The commands $\backslash$commentbill and $\backslash$commentrapiscan can be used by 
\commentbill{Bill} and \commentrapiscan{Ed/Will (or anyone else)} respectively. Alternatively one may use any software they desire, e.g. Adobe.  

\section*{Progress}

\begin{itemize}
 \item Initially started by reading Compton scattering tomography (CST) papers \cite{James2D,James3D} by James Webber.
 \item Decided to use Geant4 \cite{Geant4Bib}, successfully installed it and learnt how to compile basic demos.
 \item Understanding basics of Monte Carlo particle transport (would be nice to get this running on MATLAB too).
 \item Learnt about generating x-rays and the processes involved when coming in contact with matter (Compton/Rayleigh/Bragg/Photoelectric). Also pair production but do not know what that is.
 \item Built a basic Geant4 application : firing photons with monoenergetic energy through a Carbon solid with scattering yet to be added.
 \item Still trying to understand how to implement the scattering processes and a scoring volume. 
 \item Use the info from scoring volume to plot a histogram - just managed to install ROOT \cite{Root}.
 \item Example B4 in Geant4 manual plots a histogram using ROOT.
\end{itemize}

\section*{Questions}

\begin{itemize}
 \item How effective are the energy sensitive detectors - how near/far should they be positioned?
 \item How do we generate a spectrum of source (polyenergetic energy) and what would be a good test target to simulate?
 \item What is the proposed geometry in mind?
\end{itemize}

\section*{Next steps}

\begin{itemize}
 \item Still a novice at Geant4 so need to learn - have contacted few people in Manchester.
 \item Setup a scoring volume and implement this for monoenergetic photons to give us a histogram of the counts.
 \item Learn how to use G4Track and G4Step objects.
 \item Locate and learn how to implement scattering processes with cuts to enable switching on/off of relevant scattering processes.
 \item In this manner should have a working Geant4 code for the monoenergetic case.
 \item Experiment with the positioning of the source and detectors (change to see effect).
 \item And any other suggestions!
\end{itemize}

\bibliography{report}
\bibliographystyle{plain}

\end{document}